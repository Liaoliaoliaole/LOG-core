\subsection{Morfeas Daemon}
The Morfeas Daemon one of the Morfeas components and it's made to start each component as is specified by the user.
The specification done by the system configuration file, this file have name Morfeas\_config.xml (Example at listing \ref{lst:Morfeas_config.xml}). 
And contains the amount and type of components that consist the Morfeas system with then required arguments (or settings).\\
The construction for the Morfeas\_config.xml must done in respect to Morfeas.dtd file.

Although that the Morfeas Daemon can run in a standalone mode is more common to work together with the Systemd. 
The Systemd is a software daemon suite that developed by the Red Hat, Inc to provide an init solution for the daemons that used on the current GNU instant.
Nowadays is a standard packet in almost every distributions of the GNU operating system.

The integration procedure for Morfeas Daemon to Systemd described in details at README.md file.

When the Morfeas Daemon start the first job that is doing is to attempt a read of the Morfeas\_config.xml, after of this is check it's validity and if this is successful then 
attempt to start every component with the arguments that are given in the configuration file. For each component the Morfeas Daemon will make a new thread and will fork the component program to it.
The stdout of the component shall redirected to the Morfeas daemon's thread and from there will be redirect as reported to the systemd or if it runs in a standalone mode will print it on the display.
Also at every termination command the Morfeas Daemon have the responsibility to terminate all of it's child processes before terminate itself. 
This done by using the signaling mechanism that provided from the GNU operating system.