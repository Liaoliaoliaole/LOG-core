\section{Morfeas IPC}
The main communication way between the morfeas components is the Morfeas IPC. The Morfeas IPC is nothing more from a protocol that define the type and
the way of communication between the components. The Morfeas IPC utilities simplex communication between multiple source and one receptor.
The transaction mechanism that used for this is based on \textbf{Named Pipes} (or \textbf{FIFO}). In GNU a named pipe represent by a file,
for the Morfeas System this file located under \textbf{/tmp} and have name \textbf{Morfeas\_handlers\_FIFO}.
If the named pipe file does not exist, will be created by the first component the will not found it.

The data that transferred via the Morfeas IPC is follow the Morfeas IPC Protocol where defined at files under ``./src/IPC`` directory.
In details, the data that transmitted between the component is in a telegram form with fixed size.
The size of the telegram is defined by the message with the biggest size. The messages where does not occupied the wholly telegram space will fill the remain with zeros.
Each component use the Morfeas IPC to in three ways: To register and unregister itself, and to transfer data.
