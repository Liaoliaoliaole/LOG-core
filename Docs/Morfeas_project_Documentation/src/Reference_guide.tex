\documentclass{article}
\usepackage{ucs}
\usepackage[utf8x]{inputenc}
\usepackage[greek,english]{babel}
\usepackage{dirtree}
\usepackage{lmodern}
\usepackage[a4paper, margin=1in]{geometry}
\usepackage{enumitem}
\usepackage{xfrac}
\usepackage{lmodern}
\usepackage{amsmath,amssymb}
\usepackage[]{pdfpages}
\usepackage{multicol}
\usepackage{gensymb}
\usepackage{textcomp}
\usepackage{caption}
\usepackage{graphicx}
\usepackage[colorlinks=true,linkcolor=black]{hyperref}
\usepackage{varwidth}
\usepackage{listings}
\usepackage[bottom]{footmisc}

%Title declaration
\title{Morfeas-Core\\''Reference Guide''}
\author{Sam Harry Tzavaras}
\begin{document}
%Title page
\clearpage
\begin{figure}
\centering
  \includegraphics[width=2in]{ArtWork/Morfeas_logo.png}
\end{figure}
\maketitle
\thispagestyle{empty}
%License page
\newpage
\section{License}
Copyright (C)  12019  Sam Harry Tzavaras.\\
Permission is granted to copy, distribute and/or modify this document
under the terms of the GNU Free Documentation License, Version 1.3
or any later version published by the Free Software Foundation;
with no Invariant Sections, no Front-Cover Texts, and no Back-Cover Texts.
A copy of the license is included in the section entitled "GNU Free Documentation License".
%History page
\section{Project's History}
\date{Feb 4, 12020}:\author{Sam Harry Tzavaras} -- Initial Work.

%Contents page
\newpage
\tableofcontents
\newpage
\section{Introduction}
The Morfeas project was initially start as an implementation of a software gateway solution system
(currently named Morfeas-Proto) that provide (and translate) measurements data from some proprietary devices (SDAQ family)
with CANbus compatible interface (SDAQnet) to OPC-UA protocol (Open62541 based).

As the Morfeas project developed additional support added for other devices (MDAQ, IO-BOX, MTI) with different interfaces (ModBus-TCP, USB).

Furthermore, a web interface sub-project added to the Morfeas project under the name "Morfeas-web".
Thisof, will provide a layman friendly configuration interface for the gateway, the OPC-UA server's Nodeset and the connected devices.

\section{Installation}
To install the Morfeas Core it's required to be install the following dependencies:\\
* [GCC] - The GNU Compilers Collection\\
* [GNU Make] - GNU make utility\\
* [CMake] - Cross-platform family of open source tools for package build.\\
* [NCURSES] - A free (libre) software emulation library of curses.\\
* [GLib] - GNOME core application building blocks libraries.\\
* [LibGTop] - A library to get system specific data.\\
* [libxml2] - Library for parsing XML documents.\\
* [libmodbus] - A free software library for communication via ModBus protocol.\\
* [libi2c] - A library that provide I2C functionality to Userspace\\
* [libdbus] - The library of the D-Bus.\\
* [CAN-Utils] - CANBus utilities.\\
* [I2C-tools] - Heterogeneous set of I2C tools for Linux kernel.\\
Also the following submodules:\\
* [SDAQ worker] - A libre \(free\) utilities suite for SDAQ devices.\\
* [cJSON] - An Ultralightweight open source JSON parser for ANSI C.\\
* [open62541] - An open source C \(C99\) implementation of OPC-UA.\\

Compilation and installation instructions can be found in the README.md file at the root of source.

\section{Morfeas System Architecture}
The Morfeas system is the group of programs that implementing the operations. The design architecture that used was "Microsystem".
Multiple software that is made to doing one or more specific task(s) where communicating via Inter Processing Communication(IPC).

This design philosophy described by the name of the project ``Morfeas'' where is the is the Latinisation of of the Greek word
\selectlanguage{greek}``Μορφέας''.\selectlanguage{english} Which translate to: Him that can give or change form (or shape).

\section{Morfeas IPC}
The main communication way between the morfeas components is the Morfeas IPC. The Morfeas IPC is nothing more from a protocol that define the type and
the way of the data that transferred between the components. Is simplex with multiple source and one receptor (the transmission always start from a
Drivers/Handlers type component and end to the Morfeas OPC-UA).
The transaction mechanism that used for this is based on \textbf{Named Pipes} (or \textbf{FIFO}). In GNU (and as well in other UNIX like and based OSes) a named pipe represent by a file.
For the Morfeas System the named pipe file located under \textbf{/tmp} and have name \textbf{Morfeas\_handlers\_FIFO}.
If the named pipe file does not exist will be created by the first active component.

The data that transferred via the Morfeas IPC is follow the Morfeas IPC Protocol where defined at files under ``./src/IPC`` directory.
In details is leave from each componentin a message form with static size the maximum amount of data of the protocol. In other words the size of the data union.
Each component use the Morfeas IPC to in three ways: Register and Unregister itself and to transfer data to the Morfeas OPC-UA.
\section{Morfeas Components}
All the Morfeas components are design to run from the usespace without any special privilege.
At the following subsection will be explain in details how the implemented components works and behave.

\subsection{Morfeas Daemon}
The Morfeas Daemon is the program of the Morfeas components group where is responsible to start each component of the Morfeas system as is configured
at the Morfeas\_config.xml file. Although that the Morfeas Daemon can run standalone, is more common to work together with the Systemd daemon.
The procedure of integration of the Morfeas Daemon to the Systemd and also the configuration is described in details at README.md file at the root of source.

The responsibilities of the Morfeas Daemon is to read on start the Morfeas\_config.xml (Example at listing \ref{lst:Morfeas_config.xml}),
validate it and if the validation is successful then start each component as child process (fork). The maximum amount of components is set to 16.
The stdout of each component redirected to the Morfeas daemon and then reported to the systemd or if it runs standalone, print the report to the stdout.

\begin{lstlisting}[frame=single,caption=Example of Morfeas\_config.xml,label=lst:Morfeas_config.xml]
<?xml version="1.0" encoding="UTF-8"?>
<!DOCTYPE CONFIG SYSTEM "Morfeas.dtd">
<CONFIG>
  <CONFIGS_DIR>/home/morfeas-test/configuration</CONFIGS_DIR>
  <LOGGERS_DIR>/mnt/ramdisk/Morfeas_Loggers/</LOGGERS_DIR>
  <LOGSTAT_DIR>/mnt/ramdisk/</LOGSTAT_DIR>
  <COMPONENTS>
    <OPC_UA_SERVER><!--Can be only one-->
      <APP_NAME>Morfeas-tester(Sam)</APP_NAME>
    </OPC_UA_SERVER>
    <SDAQ_HANDLER>
      <CANBUS_IF>vcan0</CANBUS_IF>
    </SDAQ_HANDLER>
    <SDAQ_HANDLER>
      <CANBUS_IF>vcan1</CANBUS_IF>
    </SDAQ_HANDLER>
    <MDAQ_HANDLER>
      <DEV_NAME>LAB_MDAQ</DEV_NAME>
      <IPv4_ADDR>10.0.0.10</IPv4_ADDR>
    </MDAQ_HANDLER>
    <IOBOX_HANDLER>
      <DEV_NAME>LAB_IOBOX</DEV_NAME>
      <IPv4_ADDR>10.0.0.7</IPv4_ADDR>
    </IOBOX_HANDLER>
    <MTI_HANDLER>
      <DEV_NAME>LAB_MTI</DEV_NAME>
      <IPv4_ADDR>10.0.0.11</IPv4_ADDR>
    </MTI_HANDLER>
  </COMPONENTS>
</CONFIG>
\end{lstlisting}

\newpage
\subsection{Morfeas OPC-UA}
The Morfeas OPC-UA is the Morfeas component that acts as OPC-UA server of the system. The OPC-UA functionality is implemented by use of the open62541 library where is a free and open source
implementation of the OPC-UA protocol. The Morfeas OPC-UA is a multithreading program where have as main responsibility to serve the OPC-UA clients, this functionality is done at the main
thread. One other specific thread is implemented to act as Morfeas IPC receiver. This received, decoded and loaded the senders data at the Interface object of the nodeset.
The corresponding object
for each sender (Driver/Handler) component is occupied dynamically at registration.\\
Also an another tread exist with purpose to checking for changes on the OPC\_UA\_Config.xml file where is contains the ISO\_Channels configuration.
If the file found to be updated and after the a successful validation used to recreate the ISO\_Channels object.\\

An example for the OPC\_UA\_Config.xml is given at listing \ref{lst:OPC_UA_Config.xml}. 
The root node for this configuration xml file is the \textbf{NODESet}.
As childnodes is tagged as \textbf{Channel} and used in the construction of the children of ISO\_CHANNEL OPC-UA nodes. 
The childnodes of each \textbf{Channel} node guide the Morfeas OPC-UA how to handle and also give some addition information.
The content from each \textbf{ISO\_CHANNEL} node used as name for the ISO\_CHANNEL child object. 
The \textbf{INTERFACE\_TYPE} and \textbf{ANCHOR} give the information of handling and the rest(\textbf{DESCRIPTION, MIN, MAX, UNIT}) is use as static information for the ISO\_CHANNEL.

A tree view of the Morfeas OPC-UA nodeset given bellow.
\dirtree{%
 .1 /.
	.2 ISO\_Channels.
		.3 (..).
	.2 Interfaces.
		.3 IOBOX-ifs.
			.4 IOBOX-if(*).
				.5 Device\_name.
				.5 IO-BOX Status.
				.5 IO-BOX Status value.
				.5 IPv4 Address.
				.5 I-Link Power Supply.
					.6 Power Supply Vin(V).
					.6 CH(1..4).
						.7 Iout(A).
						.7 Vout(V).
				.5 Receivers.
					.6 RX(1..4).
						.7 CH(1..16).
							.8 Status.
							.8 Status\_value.
							.8 Value.
						.7 Index.
						.7 RX\_Status.
						.7 RX\_success.
		.3 MDAQ-ifs.
			.4 MDAQ-if(*).
				.5 Device Name.
				.5 IPv4 Address.
				.5 Index.
				.5 MDAQ Status.
				.5 MDAQ Status value.
				.5 Board Temperature(°C).
				.5 Channels.
					.6 (CH1..8).
						.7 Value1..3.
							.8 Measurement.
							.8 Status.
							.8 Status value.
		.3 SDAQ-ifs.
			.4 SDAQ-if(*).
				.5 Bus\_Util(\%).
				.5 Dev\_on\_Bus.
				.5 Electrics(If Morfeas\_RPI\_hat is detected).
					.6 Amperage(A).
					.6 Voltage(V).
					.6 Shunt Temp(°C).
				.5 SDAQnet.
					.6 SDAQ-*(\#\#).
						.7 Address.
						.7 Connected on Bus.
						.7 S/N.
						.7 TimeDiff.
						.7 Info.
							.8 Channels on SDAQ.
							.8 Firm\_rev.
							.8 HW\_rev.
							.8 Max\_cal\_points.
							.8 Samplerate.
							.8 Type.
						.7 Status.
							.8 Error.
							.8 Mode.
							.8 Status.
							.8 inSync.
						.7 Channels.
							.8 CH1..n.
								.9 Calibration Date.
								.9 Calibration Points.
								.9 Period (Months).
								.9 Status.
								.9 Status value.
								.9 Timestamp.
								.9 Unit.
								.9 Value.
		.3 MTI-ifs.
			.4 MTI-if(*).
				.5 Device Name.
				.5 IPv4 Address.
				.5 MTI Status.
				.5 MTI Status value.
				.5 MTI Health.
					.6 Battery Capacity(\%).
					.6 Battery Voltage(V).
					.6 Battery Status(\%).
					.6 CPU Temperature(°C).
				.5 Radio.
					.6 Data rate.
					.6 RF Channel.
					.6 Tele Dev type.
					.6 new\_MTI\_config().
					.6 Tele(Dev type depended see bellow).
						.7 **.
	.2 RPI\_Health\_status.
		.3 CPU\_Util(\%).
		.3 CPU\_Temp(°C).
		.3 Disk\_Util(\%).
		.3 RAM\_Util(\%).
}
\newpage
\begin{lstlisting}[frame=single,caption=Example of OPC\_UA\_Config.xml,label=lst:OPC_UA_Config.xml]
<?xml version="1.0" encoding="UTF-8"?>
<!DOCTYPE NODESet SYSTEM "Morfeas.dtd">
<NODESet>
  <CHANNEL>
    <ISO_CHANNEL>OT_LS790</ISO_CHANNEL>
    <INTERFACE_TYPE>SDAQ</INTERFACE_TYPE>
    <ANCHOR>1.CH1</ANCHOR>
    <DESCRIPTION>Leakage Alarm Trough B</DESCRIPTION>
    <MIN>0</MIN>
    <MAX>20</MAX>
  </CHANNEL>
  <CHANNEL>
    <ISO_CHANNEL>TE01</ISO_CHANNEL>
    <INTERFACE_TYPE>MDAQ</INTERFACE_TYPE>
    <ANCHOR>167772170.CH1.Val1</ANCHOR>
    <DESCRIPTION>Test for LAB_MDAQ CH1</DESCRIPTION>
    <MIN>0</MIN>
    <MAX>0</MAX>
    <UNIT>V</UNIT>
  </CHANNEL>
  <CHANNEL>
    <ISO_CHANNEL>TE02</ISO_CHANNEL>
    <INTERFACE_TYPE>IOBOX</INTERFACE_TYPE>
    <ANCHOR>117440522.RX1.CH2</ANCHOR>
    <DESCRIPTION>Test for LAB_IOBOX RX1 CH2</DESCRIPTION>
    <MIN>-200</MIN>
    <MAX>1200</MAX>
    <UNIT>C</UNIT>
  </CHANNEL>
</NODESet>
\end{lstlisting}

\subsection{Morfeas Drivers/Handlers}

\subsubsection{Morfeas SDAQ-if}
\subsubsection{Morfeas MDAQ-if}
\subsubsection{Morfeas IOBOX-if}
\subsubsection{Morfeas MTI-if}
\section{Integration}
\end{document}
