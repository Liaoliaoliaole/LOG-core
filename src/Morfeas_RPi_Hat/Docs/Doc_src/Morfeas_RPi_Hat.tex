\documentclass{article}
\usepackage[utf8]{inputenc}
\usepackage[a4paper, margin=1in]{geometry}
\usepackage{enumitem}
\usepackage{xfrac}
\usepackage{lmodern}
\usepackage{amsmath,amssymb}
\usepackage[]{pdfpages}
\usepackage{multicol}
\usepackage{gensymb}
\usepackage{textcomp}
\usepackage{caption}
\usepackage{graphicx}
\usepackage[colorlinks=true,linkcolor=black]{hyperref}
\usepackage{varwidth}
\usepackage{listings}
\usepackage[bottom]{footmisc}


%Commands
%text overline command
\newcommand{\textoverline}[1]{$\overline{\mbox{#1}}$}
%symbols declaration
\DeclareUnicodeCharacter{2127}{\mho}

%Title declaration
\title{Morfeas-Core\\Chapter:``Morfeas RPi Hat''}
\date{Build:\today}
\author{Sam Harry Tzavaras}

\begin{document}
%Title page
\clearpage
\begin{figure}
\centering
  \includegraphics[width=2in]{../../../Docs/Morfeas_project_Documentation/ArtWork/Morfeas_logo_green.png}
\end{figure}
\maketitle
\thispagestyle{empty}
%License page
\newpage
\section{License}
Copyright (C)  12019  Sam Harry Tzavaras.\\
Permission is granted to copy, distribute and/or modify this document
under the terms of the GNU Free Documentation License, Version 1.3
or any later version published by the Free Software Foundation;
with no Invariant Sections, no Front-Cover Texts, and no Back-Cover Texts.
A copy of the license is included in the section entitled "GNU Free Documentation License".
\section{Disclaimers}
\newpage
%Contents page
\newpage
\tableofcontents
\newpage
\section{Introduction}
This document refereed as Documentation for the "Morfeas\_RPi\_Hat" device, where is a PCB shield designed to be compatible to Raspberry Pi pin header.
The purpose of this device is to provide electrical measurements and electronic fuse for the SDAQnet port, to provide real time clock for the RPi,
and to support the backport teletype RS-323 serial port of the Morfeas system unit. The measurements that supporter is Output voltage of the port,
Supplied port's current and temperature of each current sense amplifier (CSA) silicon die.\\
In addition an electronic fuse is provided for each port that limit the supply current to $\approx$4A.
Also an EEPROM with the calibration data for each CSA is located on board. The communication between the onboard devices and the RPi done
via Inter-Integrated Circuit ($I^2C$) Bus.

\end{document}

